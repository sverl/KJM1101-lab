% !TeX encoding = UTF-8
% !TeX spellcheck = nb_NO

\documentclass[hidelinks,12pt,norsk,a4paper,fleqn]{scrartcl}

% Technical
\usepackage[T1]{fontenc}
\usepackage{comment}

% Layout
\usepackage[table]{xcolor}
\usepackage[textsize=scriptsize]{todonotes}
\usepackage{enumitem}
\usepackage{url}
\usepackage{siunitx}
\sisetup{output-decimal-marker = {,}}
\sisetup{exponent-product=\cdot}


% Tables
\usepackage{booktabs}
\usepackage{tabularx}

% Typography
\usepackage{microtype}

% Localization
\usepackage{babel}
\usepackage{isodate}
%\usepackage{icomma}

% Drawings
%\usepackage{tikz}
%\usetikzlibrary{calc}
%\usetikzlibrary{positioning}
%\usetikzlibrary{tikzmark}
%\usetikzlibrary{backgrounds}
%\newcommand{\mathtikzmark}[2]{\tikz[baseline={(#1.base)},remember picture] \node (#1) {\ensuremath{#2}};}

% Figures
\usepackage{floatrow}
\floatsetup[table]{capposition=top}

% Code
%\usepackage{minted}    
%\floatsetup[listing]{style=Plaintop}
%\AtBeginEnvironment{listing}{\setcounter{listing}{\value{lstlisting}}}
%\AtEndEnvironment{listing}{\stepcounter{lstlisting}}

% Chemistry


\usepackage{chemmacros}
\usechemmodule{reactions}

\usepackage{float}
\usepackage{varioref}
\usepackage{hyperref}
\usepackage[capitalise]{cleveref}
%\usepackage[capitalise,noabbrev]{cleveref}
\usepackage[norefs, nocites]{refcheck}
%\crefformat{footnote}{#2\footnotemark[#1]#3}



\title{Laboratorieoppgave 4\\
	   Kalorimetri og reaksjonsvarmedate}
\author{Sverre Løyland}
\date{}

\begin{document}
	
	\maketitle
	
	\section{Prelab}
	
	\subsection{Mål og hensikt}
	\begin{itemize}
		\item Forstå begrepene kalorimetri, varme, reaksjonsvarme og varmekapasitet.
		
		\item Kunne benytte lineærregresjon.
	\end{itemize}	
	
	\subsection{Teori}
	\paragraph{Temperatur}
	Temperatur er et mål på den \emph{termiske energien} til et system, forenklet kan man si at temperaturen er et mål på den gjennomsnittlige kinetiske energien i et system.
	Det vil si at i et system med høy temperatur beveger partiklene seg mye ved at de for eksempel vibrerer og roterer.
	
	\paragraph{Absolutt temperatur}
	I dagliglivet brukes vanligvis enheten grader Celsius, \si{\degreeCelsius}.
	Ordet \emph{grader} betyr at det er målt relativt til noe, for Celsius er det smelte- og kokepunktet til vann. I kjemi vil vi derimot bruke \emph{absolutte} enheter som vil si at det er målt relativt til nullpunktet.
	For temperatur betyr det at det er målt relativt til et system uten noe kinetisk energi.
	
	Den absolutte skalaen for grader Celsisus er \emph{Kelvin}.
	Legg merke til at det ikke heter \emph{grader Kelvin} siden det er absolutt skala.
	Kelvinskalaen er veldig lik skalaen for grader Celsisus, bare at den er forskjøvet med \SI{273.15}{\degreeCelsius}.
	
	\paragraph{Varme}
	I dagligtalen betyr varme det samme som temperatur, men det er ikke tilfellet i kjemi! Varme er et mål på overføring av energi mellom to systemer.
	Selv om noe har høy temperatur, betyr ikke det nødvendigvis at det overfører mye energi.
	Legg også merke til at varme overføres \emph{mellom} to systemer så det gir ikke mening i kjemien å si at \emph{én} ting har mye varme.
	
	Man har bestemt at positiv varme betyr at omgivelsene overfører energi til systemet, altså synker i temperaturen til omgivelsene, og negativ varme er at omgivelsene tar opp energi fra systemet, altså øker temperaturen til omgivelsene.
	
	\paragraph{Varmekapasitet}
	Varmekapasitet er sammenhengen mellom varme og temperaturforskjell.
	Varmekapasitet er hvor mye varme som trengs for å endre temperaturen med en viss temperaturforskjell.
	Intuitivt skjønner man at dette er avhengig av mengden stoff man har, det krever mer varme for å øke temperaturen til \SI{1}{\kilo\gram} vann enn \SI{1}{\gram} vann.
	For eksempel er varmekapasiteten til \SI{1}{\kilo\gram} vann \SI{4185.5}{\joule\per\kelvin} som vil si at det kreves \SI{4185.5}{\joule} varme for å endre temperaturen med \SI{1}{\kelvin}.
	Sammenhengen mellom varme og varmekapasitet blir da
	\begin{equation}
		q = C\Delta T \label{eq:heatcap}
	\end{equation}
	der $q$ er varmen, $C$ er varmekapasiteten og $\Delta T$ er temperaturforskjellen.
	
	\paragraph{Spesifikk og molar varmekapasitet}
	Siden varmekapasitet er avhengig av mengden stoff, oppgir man vanligvis \emph{spesifikk varmekapasitet} i stedet som er varmekapasitet per enhet masse.
	Tilsvarende forrige avsnitt er den spesifikke varmekapasiteten til vann \SI{4185.5}{\joule\per\kelvin\per\kilo\gram}.
	Sammenhengen mellom varmekapasiteten og den spesifikke varmekapasiteten blir da
	\begin{equation}
		C = cm \label{eq:specheatcap}
	\end{equation}
	der $C$ er varmekapasiteten, $c$ er den spesifikke varmekapasiteten og $m$ er massen.
	Noen ganger brukes også molar varmekapasitet som er varmekapasiteten per stoffmengde.
	
	\paragraph{Kalorimetri}
	Et kalorimeter er et system som ikke utveksler varme med omgivelsene.
	Det vil si at den totale varmeutvekslingen mellom systemet og omgivelsene er $0$, det vil si $q_\mathrm{tot}=0$.
	For eksempel kan et kalorimeter inneholde varmt og kaldt vann (som ikke er blandet).
	Når det varme og kalde vannet blander seg, vil de utveksle varme og nå likevekt.
	Varmen som er utvekslet mellom det kalde vannet og systemet med likevekt betegnes $q_\mathrm{v}$ og tilsvarende er varmen som utveksles mellom det varme vannet og systemet $q_\mathrm{k}$.
	Dersom kalorimeteret ikke utveksler noe varme, blir sammenhengen mellom $q_\mathrm{v}$ og $q_\mathrm{k}$ gitt ved
	\begin{align*}
		q_\mathrm{tot} &= 0 \\
		q_\mathrm{v} + q_\mathrm{k} &= 0.
	\end{align*}
	Setter man inn utrykkene fra \Cref{eq:heatcap} og \Cref{eq:specheatcap} får man
	\begin{equation*}
		c_{\ch{H2O}}m_\mathrm{v}(T-T_\mathrm{v}) + c_{\ch{H2O}}m_\mathrm{k}(T-T_\mathrm{k}) = 0
	\end{equation*}
		Løser man denne med hensyn på $T$ får man
	\begin{equation}
		T = \frac{m_\mathrm{v}T_\mathrm{v} + m_\mathrm{k}T_\mathrm{k}}{m_\mathrm{v} + m_\mathrm{k}}. \label{eq:Tmix}
	\end{equation}
%	Denne typen uttrykk kalles for vektede gjennomsnitt.

	Antagelsen om at kalorimeteret ikke utveksler varme med systemet er feil.
	Hvis man måler temperaturen etter blanding, vil man få et avvik fra \cref{eq:Tmix}.
	Fra dette avviket er det mulig å beregne kalorimeterets varmekapasitet.
	Dersom kalorimeteret har samme temperatur som det kalde vannet, $T_\mathrm{k}$, før det varme vannet tilsettes, er varmekapasiteten til kalorimeteret gitt ved
	\begin{equation}
		C_{\mathrm{cal}} = c_{\ch{H2O}}\left(\frac{T_\mathrm{v}-T}{T-T_\mathrm{k}}m_\mathrm{v}-m_\mathrm{k}\right) \label{eq:cal}
	\end{equation}
	der $C_{\mathrm{cal}}$ er varmekapasiteten til kalorimeteret.
	
	\paragraph{Reaksjonsvarme}
	Reaksjonsvarme er varmen som overføres i en reaksjon til systemet.
	Den molare reaksjonsvarmen er det samme som reaksjonsentalpien.
	For eksempel har reaksjonen
	\begin{reaction}
		H^+ (aq) + OH^- (aq) -> H2O (l) \label{rxn:neu}
	\end{reaction}
	en reaksjonsvarme på omtrent \SI{-60}{\kilo\joule\per\mole} reaksjonen gir fra seg energi til omgivelsene, reaksjonen er altså eksoterm.
	Denne reaksjonen kan gjøres ved å blande saltsyre, \ch{HCl (aq)}, og natriumhydroksidløsning, \ch{NaOH (aq)}.
	Da kan reaksjonsvarmen bestemmes ved å blande saltsyre og natriumhydroksidløsning med kjente volumer, konsentrasjoner og temperaturer og måle temperaturen etter endt reaksjon.
	Man kan vise at så lenge saltsyren er den begrensende reaktanten, er reaksjonsvarmen proporsjonal med volumet saltsyre.
	
	\subsection{Statistikk}
	
	\paragraph{Minste kvadraters metode}
	Ofte ønsker man å sammenligne målt data med teoretiske formler.
	Da er det vanlig at dataene ikke passer trenden til formelen perfekt så det er små avvik som kalles residualer.
	La de målte dataene være $f_i$ og de teoretiske $f(x_i)$.
	Da er residualene $r_i=f_i-f(x_i)$.
	Dersom teorien er bra, er residualene små.
	
	Det er ikke alltid man kjenner alle størrelsene i formelen $f(x)$.
	Da kan man finne disse ved å prøve forskjellige verdier for den ukjente størrelsen som minimerer residualene, dette kalles regresjon.
	Den vanligste metoden for regresjon bruker minste kvadraters metode.
	Den minimerer summen av residualene kvadrert, $\sum_{i=1}^n r_i^2 = r_1^2+r_2^2+\ldots+r_n^2$.
	\Cref{fig:lsr} viser data med lineær minste kvadraters metode og tilhørende residualer.
	
%	For lineærregresjon kan man vise at de eksplisitte uttrykkene for skjæringspunktet, $\hat{\alpha}$, og skjæringspunktet, $\hat{\beta}$, er gitt ved
%	\begin{align}
%		\hat{\alpha} &= \bar{y} - \hat{\beta}\bar{x} \\
%		\hat{\beta} &= r_{xy}\frac{s_y}{s_x} \\
%		r_{xy} &= \frac{1}{(n-1)s_xs_y}\sum_{i=1}^n(x_i-\bar{x})(y_i-\bar{y})
%	\end{align}
%	der $\bar{x}$ og $\bar{y}$ er gjennomsnittet til $x$ og $y$, $s_x$ og $s_y$ er standardavviket til $x$ og $y$ og $r_{xy}$ er den såkalte korrelasjonskoeffisienten.
%	$R^2=r_{xy}^2$ forteller noe om hvor bra regresjonslinjen passer.
%	Hvis alle regresjonslinjen $R^2=1$.

	Som nevnt er reaksjonsvarmen i \Cref{rxn:neu} proporsjonal med volumet saltsyre så lenge saltsyren er begrensende reaktant.
	Dersom man måler reaksjonsvarmen ved forskjellige volumer saltsyre, kan man bruke regresjon til å finne den molare reaksjonsvarmen siden stigningen til reaksjonsvarmen er avhengig av den molare reaksjonsvarmen.
	
	For lineærregresjon med skjæringspunktet satt til origo er stigningstallet gitt ved
	\begin{equation}
		\hat{\beta} = \frac{\sum_{i=1}^nx_iy_i}{\sum_{i=1}^nx_i^2}
	\end{equation}
	med varians
	\begin{equation}
		\sigma_{\hat{\beta}}^2 = \frac{\sum_{i=1}^{n}(y_i-\hat{\beta}x_i)^2}{(n-1)\sum_{i=1}^nx_i^2}.
	\end{equation}
	I praksis er det nesten ingen som bruker disse formlene, i stedet bruker man programpakker der formlene ligger inne eller programmerer dem.
	
	\begin{figure}[H]
		\centering
		
		% 10 random points
		\pgfmathsetseed{1}
		\pgfplotstableset{ 
			create on use/x/.style={create col/expr={\pgfplotstablerow + rand}},
			create on use/y/.style={create col/expr={(0.1*\thisrow{x}+10)+4*rand}}
		}
		\pgfplotstablenew[columns={x,y}]{10}\loadedtable
		
		% Least squares linear regression
		\pgfplotstablecreatecol[linear regression]{regression}{\loadedtable}
		
		% Residual style
		\pgfplotsset{
			residuals/.style ={
				only marks,
				error bars/y dir=minus,
				error bars/y explicit,
				/pgfplots/table/.cd,
				y error expr=\thisrow{y}-\thisrow{regression}
			}
		}
		
		\begin{tikzpicture}
			\begin{axis}[xlabel=$x$, ylabel=$y$, xmin=0, xmax=10, ymin=0, ymax=20]
				\addplot+[mark=x,red,residuals] table {\loadedtable};
%				\addplot [no markers] table [y=regression] {\loadedtable};
				\addplot[domain=0:10] {\pgfplotstableregressiona*x + \pgfplotstableregressionb};
			\end{axis}
		\end{tikzpicture}
		\caption{Datapunkter (kryss), lineær minste kvadraters regresjon (svart linje) og residualer (røde, vertikale linjer).}
		\label{fig:lsr}
	\end{figure}

	\begin{table}
		\centering
		\caption{Temperaturen $T_\mathrm{k}$ i \SI{50.0}{g} kaldt vann og temperaturen $T_\mathrm{v}$ i \SI{50.0}{g} varmt vann før blanding.
				 Etter blanding ble temperaturen ved termisk likevekt $T$.}
		\label{tbl:calexample}
		\begin{tabular}{r S[table-format=2.1] S[table-format=2.1] S[table-format=2.1] l}
			\toprule
			Replikat & $T_\mathrm{k}/\si{\degreeCelsius}$ & $T_\mathrm{v}/\si{\degreeCelsius}$ & $T/\si{\degreeCelsius}$ & $C_\mathrm{cal}/\si{\joule\per\kelvin}$ \\ \midrule
			       1 &                               21.9 &                               40.2 &                    30.9 &                     \cellcolor{blue!10} \\
			       2 &                               21.7 &                               40.2 &                    30.9 &                     \cellcolor{blue!10} \\ 
			       3 &                               22.0 &                               40.2 &                    31.0 &                     \cellcolor{blue!10} \\
			       4 &                               22.0 &                               40.2 &                    30.9 &                     \cellcolor{blue!10} \\
			       5 &                               22.2 &                               40.2 &                    31.2 &                     \cellcolor{blue!10} \\ \midrule
			                                                                            \multicolumn{4}{r}{gjennomsnitt} &                     \cellcolor{blue!10} \\
			                                                                            \multicolumn{4}{r}{standardfeil} &                     \cellcolor{blue!10} \\ \bottomrule 
		\end{tabular}
	\end{table}

	\begin{table}
		\centering
		\caption{Volumet saltsyre, $V_{\ch{HCl}}$, med temperatur $T_{\ch{HCl}}$ og temperaturen $T_{\ch{NaOH}}$ til tilsatt natriumhydroksidløsning.
				 Suren og basen hadde konsentrasjon \SI{3.000}{\molar}.
				 Etter blanding ble temperaturen ved termisk likevekt $T$.}
		\label{tbl:calexample2}
		\begin{tabular}{S[table-format=3] l S[table-format=2.1] S[table-format=2.1] S[table-format=2.1]}
			\toprule
			$V_{\ch{HCl}}/\si{\milli\liter}$ & $V_{\ch{NaOH}}/\si{\milli\liter}$ & $T_{\ch{HCl}}/\si{\degreeCelsius}$ & $T_{\ch{NaOH}}/\si{\degreeCelsius}$ & $T/\si{\degreeCelsius}$ \\ \midrule
			                              10 &               \cellcolor{blue!10} &                               22.0 &                                22.8 &                    26.7 \\
			                              20 &               \cellcolor{blue!10} &                               22.0 &                                22.7 &                    30.9 \\ 
			                              30 &               \cellcolor{blue!10} &                               22.0 &                                22.8 &                    35.3 \\
			                              40 &               \cellcolor{blue!10} &                               22.0 &                                23.0 &                    39.1 \\
			                              50 &               \cellcolor{blue!10} &                               22.0 &                                22.7 &                    42.8 \\ \bottomrule 
		\end{tabular}         
	\end{table}
	
	\subsection{Oppgaver}
	\begin{enumerate}[label=\alph*)]
		\item Utled \cref{eq:cal} ved å bruke $q_\mathrm{tot}=q_\mathrm{k} + q_\mathrm{v} + q_\mathrm{cal} = 0$, \cref{eq:heatcap} og \cref{eq:specheatcap}.
		
		\item En student helte \SI{50.0}{g} vann i et kalorimeter og målte temperaturen $T_\mathrm{k}$.
		Etter å ha tilsatt \SI{50.0}{g} varmt vann med temperatur $T_\mathrm{v}$ og blandet godt ble sluttemperaturen målt til $T$.
		Studenten gjentok forsøket 5 ganger.
		Resultatene er oppsummert i \Cref{tbl:calexample}.
		
		\item En student har et kalorimeter med saltsyre med kjent konsentrasjon, $\left[\ch{HCl}\right]$, volum, $V_{\ch{HCl}}$, og temperatur, $T_{\ch{HCl}}$.
		Til kalorimeteret tilsetter studenten en natriumhydroksidløsning med kjent konsentrasjon, $\left[\ch{NaOH}\right]$, volum, $V_{\ch{NaOH}}$, og temperatur, 	$T_{\ch{NaOH}}$ der natriumhydroksidløsningen alltid er begrensende reaktant.
		Etter en stund måler studenten temperaturen i kalorimeteret, $T$.
		\begin{enumerate}[label=\roman*)]
			\item Vis at $q_\mathrm{rxn} + q_{\ch{HCl}} + q_{\ch{NaOH}} + q_\mathrm{cal} = 0$ der $q_\mathrm{rxn}$ er reaksjonsvarmen, $q_{\ch{HCl}}$ er varmeutvekslingen mellom saltsyren og systemet i likevekt, $q_{\ch{NaOH}}$ er varmeutvekslingen mellom natriumhydroksidløsningen og systemet i likevekt og $q_\mathrm{cal}$ er varmeutvekslingen mellom kalorimeteret og systemet i likevekt.
			\item Vis at $q_\mathrm{rxn}=q_\mathrm{m, rxn} \left[\ch{HCl}\right] V_{\ch{HCl}}$ der $q_\mathrm{m, rxn}$ er den molare reaksjonsvarmen.
			\item Vis at $q_{\ch{HCl}}=c_{\ch{HCl}}\rho_{\ch{HCl}}V_{\ch{HCl}}(T - T_{\ch{HCl}})$ der $c_{\ch{HCl}}$ og $\rho_{\ch{HCl}}$ er henholdsvis den spesifikke varmekapasiteten og tettheten til saltsyren.
			\item Vis at $q_{\ch{NaOH}}=c_{\ch{NaOH}}\rho_{\ch{NaOH}}V_{\ch{NaOH}}(T - T_{\ch{NaOH}})$ der $c_{\ch{NaOH}}$ og $\rho_{\ch{HCl}}$ er henholdsvis den spesifikke varmekapasiteten og tettheten til natriumhydroksidløsningen.
			\item Vis at $q_\mathrm{cal}=C_\mathrm{cal}(T - T_{\ch{HCl}})$ der $C_\mathrm{cal}$ er varmekapasiteten til kalorimeteret.			
		\end{enumerate}
		\item Anta nå at tettheten til saltsyren og natriumhydroksidløsningen er den samme som vann, $\rho_{\ch{HCl}}=\rho_{\ch{NaOH}}=\rho_{\ch{H2O}}$, den spesifikke varmekapasiteten til saltsyren og natriumhydroksidløsningen er den samme som vann, $c_{\ch{HCl}}=c_{\ch{NaOH}}=c_{\ch{H2O}}$, det totale volumet til saltsyren og natriumhydroksidløsningen er konstant, $V_{\ch{HCl}} + V_{\ch{NaOH}} = V$, konsentrasjonene til saltsyren og natriumhydroksidløsningen er det samme, $\left[\ch{HCl}\right]=\left[\ch{NaOH}\right]$ og kalorimeteret ikke utveksler varme med systemet, $q_\mathrm{cal}=0$.
		Bruk informasjonen fra forrige oppgave til å vise at 
		\begin{equation}
			q_\mathrm{m, rxn}V_{\ch{HCl}} = \frac{c_{\ch{H2O}}\rho_{\ch{H2O}}}{\left[\ch{HCl}\right]}\left[V_{\ch{HCl}}(T_{\ch{HCl}}-T)+(V-V_{\ch{HCl}})(T_{\ch{HCl}}-T)\right] \label{eq:qmrxn}
		\end{equation}
		\item Anta videre at temperaturen til saltsyren og natriumhydroksidløsningen er den samme, $T_{\ch{HCl}}=T_{\ch{NaOH}}=T_0$, og vis at
		\begin{equation}
			\Delta T = \frac{q_\mathrm{m, rxn}\left[\ch{HCl}\right]}{c\rho V}V_{\ch{HCl}}
		\end{equation}
		der $\Delta T=T - T_0$ er temperaturforskjellen før og etter løsningene blandes.
		
		\item Vis at antagelsen om at syren er begrensende reaktant er ekvivalent med at volumet av syren er mindre enn basen når konsentrasjonene er de samme.
		
		\item En student fylte et kalorimeter med \SI{3.00}{\molar} saltsyre og målte temperaturen til $T_{\ch{HCl}}$.
		Deretter tilsatte studenten \SI{3.00}{\molar} natriumhydroksid med temperatur $T_{\ch{NaOH}}$ slik at totalvolumet var konstant og lik \SI{100}{\milli\liter}.
		Ved reaksjonsslutt ble temperaturen målt til $T$.
		Fem forsøk ble gjennomført og resultatene er oppsummert i \cref{tbl:calexample2}.
		\begin{enumerate}[label=\roman*)]
			\item Plot høyre side av \cref{eq:qmrxn} som en funksjon av tilsatt volum saltsyre, $V_{\ch{HCl}}$.
			\item Beregn den molare reaksjonsvarmen $q_\mathrm{m, rxn}$ med feilestimat ved å bruke regresjon.
		\end{enumerate}	
	\end{enumerate}
	
	\clearpage
	\section{Labøvelse}
	
	\subsection{Mål og hensikt}
	\begin{itemize}
		\item Forstå de termodynamiske begrepene kalorimetri, varmekapasitet, varmeutveksling og reaksjonsvarme.
		
		\item Bestemme den molare reaksjonsvarmen for reaksjonen mellom syre og base.
	\end{itemize}

		\subsection{Forberedelser}
	\begin{itemize}
		\item Kapittel 9.4 i læreboken.
		\item Gjennomføringen av laboratorieforsøket vises i filmen \emph{Oppgave 2 - Kalorimetri}, som du finner på \\ \url{http://www.uio.no/studier/emner/matnat/kjemi/KJM1100/h13/podcast}. \\
		Se denne før du går på lab'en.
	\end{itemize}
	
	\subsection{Eksperimentelt}
	
	\subsubsection{Sikkerhet}
	\SI{3}{\molar} salpetersyre og \SI{3}{\molar} natriumhydroksidløsning er etsende. Bruk vernebriller og labfrakk under hele forsøket.
	
	\paragraph{Ved kontakt med øynene}
	Skyll forsiktig med vann i flere minutter. Fjern eventuelle kontaktlinser dersom dette enkelt lar seg gjøre. Fortsett skyllingen.
	
	\paragraph{Ved hudkontakt}
	Vask med mye vann.
	
	\paragraph{Ved med klær}
	Skyll umiddelbart tilsølte klær og hud med mye vann før klærne fjernes. 
	
	\subsubsection{Utstyr}
	
	\begin{table}[H]
		\caption{Utstyr}
		\begin{tabular}{ll}
			\toprule
			Kalorimeter & \\
			to målesylindre & \SI{100}{\milli\liter} \\
			termometer og stoppeklokke & \\ 
			eller temperatursensor & \\
			magnetrørepinne & \\
			magnetrører & \\
			salpetersyre & \SI{3.000}{\molar} \\
			natriumhydroksidløsning & \SI{3.000}{\molar} \\
			varmebad & \\
			hansker & \\
			vernebriller & \\ 
			\bottomrule 
		\end{tabular}
		\label{equipment}
	\end{table}
	
	I \vref{equipment} er nødvendig utstyr oppgitt.	
	
	\subsubsection{Fremgangsmåte}
	\begin{enumerate}
		\item
		\begin{enumerate}
			\item Sett kalorimeteret oppå magnetrøreren og legg magnetrørepinnen oppi.  Hell \SI{10}{\milli\liter} \SI{3.000}{\molar} salpetersyre i kalorimeteret og sett på lokket. Start magnetrøreren. \label{lst:firststep}
			
			\item Sett termometeret eller temperatursensoren nedi lokket. Pass på at du ikke setter det så hardt nedi at det blir hull i bunnen av kalorimeteret. Mål temperaturen etter \SI{2}{\minute}.
			
			\item Mål temperaturen til \SI{90}{\milli\liter} \SI{3.000}{\molar} salpetersyre og tilsett den til kalorimeteret. Mål temperaturen etter \SI{2}{\minute}. \label{lst:thirdstep}
			
			\item Gjenta \crefrange{lst:firststep}{lst:thirdstep} for volumene i \Cref{tbl:cal}.
		\end{enumerate}
	
		\item
		\begin{enumerate}
			\item 
		\end{enumerate}	
	\end{enumerate}

		\begin{table}
		\centering
		\caption{Volumet salpetersyre, $V_{\ch{HNO3}}$, med temperatur $T_{\ch{HNO3}}$ og volumet natriumhydroksidløsning, $V_{\ch{NaOH}}$, med temperatur $T_{\ch{NaOH}}$.
			Syren og basen hadde konsentrasjon \SI{3.000}{\molar}.
			\SI{2}{\minute} etter blanding ble temperaturen $T$ målt.}
		\label{tbl:cal}
		\begin{tabular}{S[table-format=2]llll}
			\toprule
			$V_{\ch{HCl}}/\si{\milli\litre}$ & $V_{\ch{NaOH}}/\si{\litre}$ & $T_{\ch{HCl}}/\si{\degreeCelsius}$ & $T_{\ch{NaOH}}/\si{\degreeCelsius}$ & $T/\si{\degreeCelsius}$ \\ \midrule
			                              10 &                          90 &                \cellcolor{blue!10} &                 \cellcolor{blue!10} &     \cellcolor{blue!10} \\
			                              20 &                          80 &                \cellcolor{blue!10} &                 \cellcolor{blue!10} &     \cellcolor{blue!10} \\ 
			                              30 &                          70 &                \cellcolor{blue!10} &                 \cellcolor{blue!10} &     \cellcolor{blue!10} \\
			                              40 &                          60 &                \cellcolor{blue!10} &                 \cellcolor{blue!10} &     \cellcolor{blue!10} \\
			                              50 &                          50 &                \cellcolor{blue!10} &                 \cellcolor{blue!10} &     \cellcolor{blue!10} \\ \bottomrule 
		\end{tabular}         
	\end{table}	
	
	% !TeX spellcheck = nb_NO

\section{Rapport}
Øverst på rapporten skal det stå:
\begin{itemize}
	\item Dato for gjennomføring av labøvelsen.
	\item Navn og plassnummer på laboratoriet.
	\item Navn på labpartner.
	\item Gruppenummer.
	\item Gruppelærer.
\end{itemize}

Rapporten skal inneholde:
\paragraph{Hensikt}
Forklar \emph{kort} hensikten og målet med øvelsen med dine egne ord.

\paragraph{Eksperimentelt}
Forklar \emph{kort} den eksperimentelle delen med egne ord.

\paragraph{Resultater}
Oppsummer målingene dine i tabeller.

Beregn gjennomsnitt, standardavvik og relativt standardavvik for pipetten, målesylinderen og målebegeret.

\paragraph{Diskusjon}
\begin{itemize}
	\item Vurder resultatene dine. 
	
	\item Sammenlign nøyaktigheten og presisjonen til det forskjellige utstyret. Hva finner du? 
	
	\item Stemmer resultatene overens med forventningene?
	
	\item Skjedde det endringer eller feil under forsøket som kan ha påvirket resultatet?
\end{itemize}
	
\end{document}