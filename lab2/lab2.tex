% !TeX encoding = UTF-8
% !TeX spellcheck = nb_NO

\documentclass[hidelinks,12pt,norsk,a4paper,fleqn]{scrartcl}

% Technical
\usepackage[T1]{fontenc}
\usepackage{comment}

% Layout
\usepackage[table]{xcolor}
\usepackage[textsize=scriptsize]{todonotes}
\usepackage{enumitem}
\usepackage{url}
\usepackage{siunitx}
\sisetup{output-decimal-marker = {,}}
\sisetup{exponent-product=\cdot}


% Tables
\usepackage{booktabs}
\usepackage{tabularx}

% Typography
\usepackage{microtype}

% Localization
\usepackage{babel}
\usepackage{isodate}
%\usepackage{icomma}

% Drawings
%\usepackage{tikz}
%\usetikzlibrary{calc}
%\usetikzlibrary{positioning}
%\usetikzlibrary{tikzmark}
%\usetikzlibrary{backgrounds}
%\newcommand{\mathtikzmark}[2]{\tikz[baseline={(#1.base)},remember picture] \node (#1) {\ensuremath{#2}};}

% Figures
\usepackage{floatrow}
\floatsetup[table]{capposition=top}

% Code
%\usepackage{minted}    
%\floatsetup[listing]{style=Plaintop}
%\AtBeginEnvironment{listing}{\setcounter{listing}{\value{lstlisting}}}
%\AtEndEnvironment{listing}{\stepcounter{lstlisting}}

% Chemistry


\usepackage{chemmacros}
\usechemmodule{reactions}

\usepackage{float}
\usepackage{varioref}
\usepackage{hyperref}
\usepackage[capitalise]{cleveref}
%\usepackage[capitalise,noabbrev]{cleveref}
\usepackage[norefs, nocites]{refcheck}
%\crefformat{footnote}{#2\footnotemark[#1]#3}



\title{Laboratorieoppgave 2:\\
	Ideelle gasser og \\
	magnesiums molarmasse}
\author{}
\date{}

\begin{document}
			
	\maketitle
	
	\section{Prelab}
	
	\subsection{Mål og hensikt}
	\begin{itemize}
		\item Vite hva en ideell gass er og skjønne hvordan den oppfører seg ut fra intuisjon og ligninger
		
		\item Finne molarmassen til et metall ved å bruke teori om ideelle gasser.
		
		\item Kunne vurdere presisjon og nøyaktighet av data ved å tolke beregnet konfidensintervall.
	\end{itemize}
	
	\subsection{Statistikk}
	\paragraph{Konfidensintervall}
	I laboppgave 1 ble standardavvik brukt som mål på spredningen til målingene. Et annet statistisk mål som ofte brukes er konfidensintervall som angir et intervall med verdier som er gode anslag til den faktiske verdien. 
	
	\begin{comment}
		I laboppgave 1 ble resulatet oppgitt med $\text{gjennomsnittet} \pm \text{standardavviket}$, altså $\bar{x}\pm s$. I eksempelet ble det \SI[separate-uncertainty=true]{2.94 +- 0.12}{g}. Dette er en type konfidensintervall med \SI{68}{\percent} konfidens\todo{Kommentar om hvordan konfidensintervall skrives?}. Det vil si at hvis man hadde gjentatt forsøket mange ganger og regnet ut konfidensintervallene, ville \SI{68}{\percent} av intervallene inneholdt den faktiske verdien. 
	\end{comment}
	
	I laboppgave 1 ble resulatet oppgitt med $\text{gjennomsnittet} \pm \text{standardavviket}$, altså $\bar{x}\pm s$. I eksempelet ble det \SI[separate-uncertainty=true]{2.94 +- 0.12}{g}. Skrevet som et intervall er det $(\SI{2.82}{g}; \SI{3.06}{g})$. Dette er et intervall som antyder en viss sikkerhet til målingene. Konfidensintervall er et lignende intervall som angis med en viss konfidens. Konfidensnivået er andelen av intervallene som ville inneholdt den faktiske verdien dersom forsøket hadde blitt gjentatt mange ganger.
	
	For å regne ut konfidensintervallet til $n$ standardfordelte målinger må gjennomsnittet $\bar{x}$ og standardavviket $s$ først regnes ut. Det er ikke pensum å vite hva standardfordelingen er og hvordan man kan undersøke om målinger er standardfordelte, men de fleste målingene i labkurset er tilnærmet normalfordelte. Da er konfidensintervallet gitt ved
	\begin{equation*}
		\left(\bar{x}-t^*\frac{s}{\sqrt{n}}; \bar{x}+t^*\frac{s}{\sqrt{n}}\right)
	\end{equation*}
	der $t^*$ er avhengig av $n$ og konfidensnivået og må slåes opp i en tabell eller regnes ut av programvare. \vref{ttable} gir $t^*$-verdier for en rekke konfidensnivåer. Kolonnen df står for \emph{degrees of freedom} og er i dette tilfellet $n-1$. 
	
	I eksempelet i laboppgave 1 var det fire målinger. Gjennomsnittet og standardavviket var henholdsvis \SI{2.94}{g} og \SI{0.12}{g}. Siden det var fire målinger er antall frihetsgrader tre. Da er $t^*$-verdien for et \SI{95}{\percent} konfidensintervall \num{3.182}. Da blir konfidensintervallet
	\begin{align*}
		&\left(\bar{x}-t^*\frac{s}{\sqrt{n}}; \bar{x}+t^*\frac{s}{\sqrt{n}}\right)\\
		=&\left(\SI{2.94}{g}-\num{3.182}\cdot\frac{\SI{0.12}{g}}{\sqrt{4}}; \SI{2.94}{g}+\num{3.182}\cdot\frac{\SI{0.12}{g}}{\sqrt{4}}\right)\\
		=&\left(\SI{2.749}{g}; \SI{3.131}{g}\right)
	\end{align*} 
	
	\begin{table}[H]
		\centering
		\caption{Tabell for $t^*$-verdier for konfidensintervall. Verdiene i eksempelet er uthevet.}
		\label{ttable}
		\begin{tabular}{rS[table-format=1.3]S[table-format=2.3]S[table-format=2.3]S[table-format=3.3]}
			\toprule
			& \multicolumn{4}{c}{Konfidensnivå} \\ \cmidrule(lr){2-5}
			df  & \SI{90}{\percent} & \cellcolor{blue!20} \SI{95}{\percent} & \SI{99}{\percent}  & \SI{99.9}{\percent} \\ \midrule
			1   & 6.314 & \cellcolor{blue!10}12.710 & 63.660 & 636.62 \\
			2   & 2.920 & \cellcolor{blue!10}4.303  & 9.925  & 31.599 \\
			\cellcolor{blue!20} 3   & \cellcolor{blue!10} 2.353 & \cellcolor{blue!30} 3.182  & 5.841  & 12.924 \\
			4   & 2.132 & 2.776  & 4.604  & 8.610  \\
			5   & 2.015 & 2.571  & 4.032  & 6.869  \\
			6   & 1.943 & 2.447  & 3.707  & 5.959  \\
			7   & 1.895 & 2.365  & 3.499  & 5.408  \\
			8   & 1.860 & 2.306  & 3.355  & 5.041  \\
			9   & 1.833 & 2.262  & 3.250  & 4.781  \\
			10  & 1.812 & 2.228  & 3.169  & 4.587  \\
			15  & 1.753 & 2.131  & 2.947  & 4.073  \\
			20  & 1.725 & 2.086  & 2.845  & 3.850  \\
			29  & 1.699 & 2.045  & 2.756  & 3.659  \\
			30  & 1.697 & 2.042  & 2.750  & 3.646  \\
			40  & 1.684 & 2.021  & 2.704  & 3.551  \\
			60  & 1.671 & 2.000  & 2.660  & 3.460  \\
			80  & 1.664 & 1.990  & 2.639  & 3.416  \\
			100 & 1.660 & 1.984  & 2.626  & 3.390  \\
			$\infty$   & 1.645 & 1.960  & 2.576  & 3.291  \\ \bottomrule
		\end{tabular}
	\end{table}	
		
	\subsection{Teori}
	\paragraph{Reaksjonen}
	I denne øvelsen vil reaksjonen mellom et metall og en syre bli studert. For eksempel vil sink og saltsyre reagere og danne hydrogengass etter reaksjonslikningen
	\begin{equation*}
		\ch{Zn (s) + 2 HCl (aq) -> ZnCl2 (aq) + H2 (g)}.
	\end{equation*}
	Dersom en kjent masse av metallet reagerer fullstendig med syren og man samler opp hydrogengassen som dannes, kan man bestemme molarmassen $M$ til metallet ved likningen
	\begin{equation*}
		M = \frac{m}{n}
	\end{equation*}
	der $m$ og $n$ er henholdsvis massen og stoffmengden til metallet. Stoffmengden av metallet kan relateres til stoffmengden av hydrogengassen ved hjelp av reaksjonslikningen. 
	
	\paragraph{Ideelle gasser}
	Stoffmengden hydrogen kan tilnærmet bestemmes fra den ideelle gassloven
	\begin{equation*}
		PV = RnT
	\end{equation*}
	der $P$, $V$, $n$ og $T$ er henholdsvis trykket, volumet, stoffmengden og temperaturen til hydrogengassen og $R$ er gasskonstanten. Volumet og temperaturen kan måles direkte, men trykket er litt mer innviklet å finne. 
	
	\paragraph{Trykk}
	Beholderen som vil bli fylt med hydrogengass i forsøket er et langt, tynt målerør som står vertikalt og er i utgangpunktet fylt med vann. Når reaksjonen skjer vil hydrogengassen boble opp i røret og fortrenge en del av vannet slik at det er gassen legger seg øverst i røret og resten er fylt med vann. 
	
	Summen av trykkene inni røret med vann og gass må være likt det ytre, atmosfæriske trykket. 
	
	Trykket i en væske på dypde en $h$ er gitt ved 
	\begin{equation}
		P=\rho gh \label{eq:rgh}
	\end{equation}
	der $\rho$ er tettheten til væsken og $g$ er tyngdeakselerasjonen.
	
	\paragraph{Damptrykk}
	Alle væsker fordamper litt ved alle temperaturer, ikke bare ved kokepunktet. Hvor mye den fordamper kalles væskens damptrykk. Sammen med hydrogengassen vil det derfor også være litt vanndamp som vil bidra til trykket.
	
	Damptrykket til vann kan for eksempel tilnærmes ved lineær interpolasjon. Interpolasjon er når man kjenner enkelte datapunkter $(x_1, y_1), (x_2, y_2),\ldots, (x_n, y_n)$ og vil finne en tilnærming til en $y$-verdi gitt en $x$-verdi. I lineær interpolasjon gjøres dette ved å anta verdiene endres lineært mellom de kjente verdiene. Gitt en $x$-verdi mellom $x_i$ og $x_{i+1}$ er den tilnærmede $y$-verdien gitt ved
	\begin{equation}
		y = y_i + \frac{y_{i+1} - y_i}{x_{i+1} - x_i}(x-x_i). \label{eq:inter}
	\end{equation}
	
	\begin{comment}
		Gitt en $x$-verdi mellom $x_i$ og $x_{i+1}$ er stigningen mellom punktene gitt ved
		\begin{align*}
		a &= \frac{\Delta y}{\Delta x}\\
		&= \frac{y_{i+1} - y_i}{x_{i+1} - x_i}. 
		\end{align*}
		Den rette linjen mellom punktene kan da finnes fra ettpunktsformelen ved
		\begin{align*}
		y-y_i &= a (x-x_i) \\
		y &= y_i + a (x-x_i) \\
		&= y_i + \frac{y_{i+1} - y_i}{x_{i+1} - x_i}(x-x_i).
		\end{align*}
	\end{comment}
	
	\Vref{vap} gir vannets damptrykk ved utvalgte temperaturer og fra den kan flere verdier interpoleres.
	
	\begin{table}[H]
		\centering
		\caption{Vannets damptrykk $P$ ved utvalgte temperaturer $T$}
		\label{vap}
		\begin{tabular}{S[table-format=2.1]S[table-format=1.5]}
			\toprule
			$T/\si{\celsius}$ & $P/\si{atm}$ \\ \midrule
			16.0 & 0.01793 \\
			18.0 & 0.02036 \\
			20.0 & 0.02308 \\
			22.0 & 0.02609 \\
			24.0 & 0.02945 \\
			26.0 & 0.03317 \\
			28.0 & 0.03730 \\
			30.0 & 0.04187 \\ \bottomrule
		\end{tabular}
	\end{table}
	
	\subsection{Oppgaver}
	\begin{enumerate}[label=\alph*)]
		\item Hvilke antagelser brukes i den ideelle gasslikningen.
		
		\item Utled \vref{eq:rgh} ved å anta at søylen har tverrsnitt $A$ og høyde $h$. Bruk at trykket er gitt ved $P=F/A$ der $F$ er tyngdekraften til vannet.	
		
		\item Utled \vref{eq:inter} ved å finne stigningen mellom punktene $(x_i, y_i)$ og $(x_{i+1}, y_{i+1})$ og for eksempel bruke ettpunktsformelen. 
		
		\item I et førsøk skulle molarmassen til sink bestemmes. \SI{0.090}{g} sink ble veid ut og reagert med saltsyre som produserte hydrogengass som ble samlet opp i et målerør.
		\begin{enumerate}[label=\roman*)]
			\item Høyden av vannet i målerøret ble målt til \SI{135.0}{mm}. Regn ut trykket av vannet $P_w$.
			
			\item Temperaturen i rommet ble målt til \SI{23.2}{\celsius}. Beregn vannets damptrykk $P_\mathrm{vap}$ inne i målerøret.
			
			\item Barometertrykket $P_0$ (dvs. det ytre trykket) ble målt til $P_0=\SI{1.013}{atm}$. Beregn trykket til hydrogengassen inni målesylinderen. Bruk at summen av trykkene utenfor og inni målesylinderen må være det samme.
			
			\item Regn ut stoffmengden hydrogengass. 
			
			\item Regn ut molarmassen til sink.
			
			\item \SI{10}{mL} \SI{6}{M} saltsyre ble brukt. Var det tilstrekkelig med syre?
		\end{enumerate}
	
		\item Et annet forsøk for å bestemme molarmassen til sink ble repetert fem ganger og replikatene ga molarmassene \SI{65,40}{g/mol}, \SI{66,56}{g}, \SI{65,60}{g}, \SI{64,94}{g} og \SI{65,19}{g}.
		\begin{enumerate}[label=\roman*)]
			\item Beregn den gjennomsnittlige molarmassen.
			
			\item Beregn standardavviket for molarmassen.
			
			\item Beregn det relative standardavviket for molarmassen.
			
			\item Beregn et \SI{95}{\percent} konfidensintervall for molarmassen.
		\end{enumerate}
		
	\end{enumerate}
	
	\clearpage
	\section{Labøvelse}
	
	\subsection{Mål og hensikt}
	\begin{itemize}
		\item Bestemme molarmassen til magnesium.
		
		\item Vurdere presisjon og nøyaktighet av data ved å tolke beregnet gjennomsnitt, standardavvik og konfidensintervall.
	\end{itemize}
	
	\subsection{Eksperimentelt}
	
	\subsubsection{Utstyr}
	
	\begin{table}[H]
		\caption{Utstyr}
		\begin{tabular}{ll}
			\toprule
			magnesiumbånd & \\ 
			saltsyre & (\SI{6}{M}) \\
			smergelpapir & \\
			målerør & \\
			slangebit & \\
			hansker & \\
			spruteflaske & (ionebyttet vann) \\
			begerglass & (\SI{150}{mL}) \\
			kobbertråd & \\
			avbitertang & \\ 
			stativ med klemme & \\ 
			vekt & (felles) \\ \bottomrule 
		\end{tabular}
		\label{equipment}
	\end{table}

	I \vref{equipment} er nødvendig utstyr oppgitt.	Noter hvilken vekt du benytter og toleransen dens. Du burde bruke samme vekt til alle målingene dine for at de skal være konsekvente.
	
	\subsubsection{Fremgangsmåte}
	\begin{itemize}
		\item Puss et magnesiumbånd med smergelpapir og tørk det for å fjerne mulig oksidlag.
		
		\item Klipp båndet i fem omtrent like store biter med avbitertang.
		
		\item Vei magnesiumbitene. Pass på at du har kontroll på hvilke biter som veier hva.
		
		\item Følgende del repeteres for hver magnesiumbit:
		\begin{itemize}
			\item Vikle den ene enden av kobbertråden rundt magnesiumbiten. 
			
			\item Hell ca. \SI{10}{mL} \SI{6}{M} saltsyre i målerøret. Fyll resten av røret forsiktig med ionebyttet vann så den tyngre saltsyren ligger i bunn med vannet oppå.
			
			\item Heng magnesiumbiten med kobbertråden ca. \SI{5}{cm} ned i målerøret og fest tråden ved å sette en slangebit i åpningen. Fyll opp med ionebyttet vann.
			
			\item Fyll begerglasset halvfullt med vann og sett opp et stativ med klemme ved siden av. Hold over åpningen i målerøret og snu det opp ned og sett den ned i begerglasset. Fest målerøret i stativet med klemmen.
			
			\item Saltsyren burde strømme ned og reagere med magnesiumbiten så det produseres hydrogengass som samles i målerøret. Vent til reaksjonen er ferdig.
			
			\item Mål temperaturen i nærheten av målerøret, mål volumet av gassen og mål vannhøyden fra overflaten i begerglasset til bunnen av menisken i målerøret.
		\end{itemize}
	
		\item En veileder vil oppgi barometertrykket.
		
	\end{itemize}	
	
	% !TeX spellcheck = nb_NO

\section{Rapport}
Øverst på rapporten skal det stå:
\begin{itemize}
	\item Dato for gjennomføring av labøvelsen.
	\item Navn og plassnummer på laboratoriet.
	\item Navn på labpartner.
	\item Gruppenummer.
	\item Gruppelærer.
\end{itemize}

Rapporten skal inneholde:
\paragraph{Hensikt}
Forklar \emph{kort} hensikten og målet med øvelsen med dine egne ord.

\paragraph{Eksperimentelt}
Forklar \emph{kort} den eksperimentelle delen med egne ord.

\paragraph{Resultater}
Oppsummer målingene dine i tabeller.

Beregn gjennomsnitt, standardavvik og relativt standardavvik for pipetten, målesylinderen og målebegeret.

\paragraph{Diskusjon}
\begin{itemize}
	\item Vurder resultatene dine. 
	
	\item Sammenlign nøyaktigheten og presisjonen til det forskjellige utstyret. Hva finner du? 
	
	\item Stemmer resultatene overens med forventningene?
	
	\item Skjedde det endringer eller feil under forsøket som kan ha påvirket resultatet?
\end{itemize}
	
	
\end{document}