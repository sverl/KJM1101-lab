% !TeX encoding = UTF-8 
% !TeX spellcheck = nb_NO 

\documentclass[hidelinks,12pt,norsk,a4paper,fleqn]{scrartcl}

% Technical
\usepackage[T1]{fontenc}
\usepackage{comment}

% Layout
\usepackage[table]{xcolor}
\usepackage[textsize=scriptsize]{todonotes}
\usepackage{enumitem}
\usepackage{url}
\usepackage{siunitx}
\sisetup{output-decimal-marker = {,}}
\sisetup{exponent-product=\cdot}


% Tables
\usepackage{booktabs}
\usepackage{tabularx}

% Typography
\usepackage{microtype}

% Localization
\usepackage{babel}
\usepackage{isodate}
%\usepackage{icomma}

% Drawings
%\usepackage{tikz}
%\usetikzlibrary{calc}
%\usetikzlibrary{positioning}
%\usetikzlibrary{tikzmark}
%\usetikzlibrary{backgrounds}
%\newcommand{\mathtikzmark}[2]{\tikz[baseline={(#1.base)},remember picture] \node (#1) {\ensuremath{#2}};}

% Figures
\usepackage{floatrow}
\floatsetup[table]{capposition=top}

% Code
%\usepackage{minted}    
%\floatsetup[listing]{style=Plaintop}
%\AtBeginEnvironment{listing}{\setcounter{listing}{\value{lstlisting}}}
%\AtEndEnvironment{listing}{\stepcounter{lstlisting}}

% Chemistry


\usepackage{chemmacros}
\usechemmodule{reactions}

\usepackage{float}
\usepackage{varioref}
\usepackage{hyperref}
\usepackage[capitalise]{cleveref}
%\usepackage[capitalise,noabbrev]{cleveref}
\usepackage[norefs, nocites]{refcheck}
%\crefformat{footnote}{#2\footnotemark[#1]#3}

 

\title{Laboratorieoppgave 2:\\
	Ideelle gasser og den \\
	molare massen til magnesium}
\author{Sverre Løyland og Karoline Fægri}
\date{}

\begin{document}
	
	\maketitle
	
	\section{Prelab}
	
	\subsection{Mål og hensikt}
	\begin{itemize}
		\item Vite hva en ideell gass er og skjønne hvordan den påvirkes av omgivelsene.
		
		\item Finne den molare massen til et metall ved å bruke teori om ideelle gasser.
		
		\item Kunne beregne usikkerhet uttrykt som standardfeil. 
	\end{itemize}
	
	\subsection{Forberedelser}
	\begin{itemize}
		\item Kapittel 5.1-5.5 i læreboken
		
		\item Gjennomføringen av laboratorieforsøket vises i filmen Oppgave 1A - magnesium, som du finner på \\ \url{http://www.uio.no/studier/emner/matnat/kjemi/KJM1100/h13/podcast}. \\
		Se denne før du går på lab'en.
	\end{itemize}
	
	\subsection{Statistikk}
	\paragraph{Standardfeil}
	Standardfeilen, som også kalles standardavviket for middelverdien/gjennomsnittet, er et mål på usikkerheten som kommer av at vi bruker vårt bergnede gjennomsnitt istedenfor det sanne gjennomsnittet. Standardfeilen betegnes ofte SE (for standard error).
	
	Jo flere målinger vi gjør, dess mer representativt er gjennomsnittet vårt for det sanne gjennomsnittet, og jo mindre spredningen i målingene er, desto mer sannsynlig er det at gjennomsnittet ligger nær den "sanne" verdien. Standardfeilen er med andre ord avhengig både av antall målinger og spredningen i målingene, og uttrykkes som 
	\begin{equation}
		SE =\frac{s}{\sqrt{n}}
	\end{equation}
	
	Det er vanlig å oppgi gjennomsnittet med usikkerhet uttrykt som standardfeil, altså $\bar{x}\pm SE$.
	
	
	Vær oppmerksom på at usikkerhet beregnet som standardfeil antar at målingene er normalfordelte og at det bare er tilfeldige feil. Ingen vanlige statistiske metoder tar høyde for systematiske feil. Den eneste måten å unngå systematiske feil på, er å jobbe med riktig teknikk og sørge for at utstyret du bruker er kalibrert.
	
	
	\subsection{Teori}
	\paragraph{Reaksjonen}
	Mange metaller reagerer med syrer. For eksempel vil sink og saltsyre reagere og danne hydrogengass etter reaksjonslikningen
	\begin{equation*}
		\ch{Zn (s) + 2 HCl (aq) -> ZnCl2 (aq) + H2 (g)}    
	\end{equation*}
	
	Hvis en kjent masse av metallet reagerer fullstendig med syren og man samler opp hydrogengassen som dannes, kan man bruke gassvolumet til å finne stoffmengden hydrogengass (se neste avsnitt), og ved hjelp av reaksjonsligningen beregne stoffmengden (antall mol) av metallet. Da kan man bestemme den molare massen $M$ til metallet ved likningen
	\begin{equation}
	M = \frac{m}{n}
	\end{equation}
	der $m$ og $n$ er henholdsvis massen og stoffmengden til metallet. Stoffmengden av metallet kan relateres til stoffmengden av hydrogengassen ved hjelp av reaksjonslikningen. 
	
	\paragraph{Ideelle gasser}
	Stoffmengden hydrogen kan tilnærmet bestemmes fra den ideelle gassloven
	\begin{equation}
	PV = nRT
	\end{equation}
	der $P$, $V$, $n$ og $T$ er henholdsvis trykket, volumet, stoffmengden og temperaturen til hydrogengassen og $R$ er gasskonstanten. Volumet og temperaturen kan måles direkte, men trykket er litt mer innviklet å finne. 
	
	\paragraph{Trykk}
	Beholderen som skal samle opp hydrogengass i forsøket, er et langt, tynt målerør som står vertikalt. Røret er i utgangpunktet fylt med vann. Når reaksjonen skjer, vil hydrogengassen boble opp i røret og fortrenge en del av vannet slik at gassen fyller den øvre delen av røret. 
	
	Summen av trykkene inni røret med vann og gass må være likt det ytre trykket. 
	
	Trykket i en væskesøyle med høyde $h$ over væskeflaten er gitt ved 
	\begin{equation}
	P=\rho gh \label{eq:rgh}
	\end{equation}
	der $\rho$ er tettheten til væsken og $g$ er tyngdeakselerasjonen.
	
	\paragraph{Damptrykk}
	Alle væsker fordamper litt ved alle temperaturer, ikke bare ved kokepunktet. Hvor mye den fordamper kalles væskens damptrykk. Sammen med hydrogengassen vil det derfor også være litt vanndamp som vil bidra til trykket inne i målerøret.
	
	Tabell \ref{vap} gir vannets damptrykk ved utvalgte temperaturer. For å finne damptrykket for temperaturer som ikke står i tabellen, kan man gjøre en lineær interpolasjon. Lineær interpolasjon kan du bruke når måleverdiene endrer seg lineært. Da kan du anslå verdien $y$ i et punkt $x$ mellom to punkter $x_i$ og $x_{i+1}$  ved å bruke formelen
	\begin{equation}
	y = y_i + \frac{y_{i+1} - y_i}{x_{i+1} - x_i}(x-x_i). \label{eq:inter}
	\end{equation}
	
	
	
	\begin{table}[htpb]
		\centering
		\caption{Vannets damptrykk $P$ ved utvalgte temperaturer $\Theta$}
		\label{vap}
		\begin{tabular}{rr}
			\toprule
			$\Theta/\si{\degreeCelsius}$ & $P/\text{atm}$ \\
			\midrule
			16.0 & 0.01793 \\
			18.0 & 0.02036 \\
			20.0 & 0.02308 \\
			22.0 & 0.02609 \\
			24.0 & 0.02945 \\
			26.0 & 0.03317 \\
			28.0 & 0.03730 \\
			30.0 & 0.04187 \\ \bottomrule
		\end{tabular}
	\end{table}
	
	\subsection{Oppgaver}
	\begin{enumerate}
		\item Hvilke antagelser brukes i den ideelle gasslikningen.
		\item Utled ligning \eqref{eq:rgh} ved å anta at søylen har tverrsnitt $A$ og høyde $h$. Bruk at trykket er gitt ved $P=F/A$ der $F$ er tyngdekraften til vannet.	
		\item Utled ligning  \eqref{eq:inter} ved å finne stigningen mellom punktene $(x_i, y_i)$ og $(x_{i+1}, y_{i+1})$ og for eksempel bruke ettpunktsformelen. Du kan lese om ettpunktsformelen på
		
		https://www.matematikk.org/oss.html?tid=89173
		\item I et førsøk skulle den molare massen til sink bestemmes. \SI{0.090}{g} sink ble veid ut og tilsatt saltsyre. Reaksjonen produserte hydrogengass, som ble samlet opp i et målerør.
		\begin{enumerate}
			\item Høyden av vannet i målerøret ble målt til \SI{135.0}{mm}. Regn ut trykket av vannet $P_w$.
			\item Temperaturen i rommet ble målt til \SI{23.2}{\degreeCelsius}. Beregn vannets damptrykk $P_\mathrm{vap}$ inne i målerøret.
			\item Barometertrykket $P_0$ (dvs. det ytre trykket) ble målt til $P_0=\SI{1.013}{atm}$. Beregn trykket til hydrogengassen inni målesylinderen. Bruk at summen av trykkene utenfor og inni målesylinderen må være det samme.
			\item Avlest volum hydrogengass i forsøket var 34,77 mL. Regn ut stoffmengden hydrogengass.
			\item Regn ut den molare massen til sink.
			\item \SI{10}{mL} \SI{6}{M} saltsyre ble brukt. Var det tilstrekkelig med syre?
		\end{enumerate}
		\item Et annet forsøk for å bestemme molarmassen til sink ble utført fem ganger og replikatene ga molarmassene \SI{65,40}{g/mol}, \SI{66,56}{g/mol}, \SI{65,60}{g/mol}, \SI{64,94}{g/mol} og \SI{65,19}{g/mol}.
		\begin{enumerate}
			\item Beregn den gjennomsnittlige molare massen.
			\item Beregn standardavviket for den molare massen.
			\item Beregn det relative standardavviket for den molare massen.
			\item Beregn standardfeielen for den molare massen.
			\item Oppgi den molare massen til sink med usikkerhet i form av standardfeil. Hvordan stemmer resultatet med den offisielle molare massen til sink, som er \SI{65.38}{g/mol}.
		\end{enumerate}
	\end{enumerate}
	
	\clearpage
	
	\section{Labøvelse}
	
	\subsection{Mål og hensikt}
	\begin{itemize}
		\item Bestemme den molare massen til magnesium.
		\item Vurdere presisjon og nøyaktighet av data ved å tolke beregnet gjennomsnitt, standardavvik og standardfeil.
	\end{itemize}
	
	\subsection{Eksperimentelt}
	
	\subsubsection{Sikkerhet}
	\SI{6}{M} \ch{HCl} (saltsyre) er etsende. Bruk vernebriller og labfrakk under hele forsøket.
	
	VED KONTAKT MED ØYNENE: Skyll forsiktig med vann i flere minutter. Fjern eventuelle kontaktlinser dersom dette enkelt lar seg gjøre. Fortsett skyllingen.
	
	VED HUDKONTAKT: Vask med mye vann
	
	VED KONTAKT MED KLÆR: Skyll umiddelbart tilsølte klær og hud med mye vann før klærne fjernes. 
	
	\subsubsection{Utstyr}
	
	\begin{table}[htpb]
		\caption{Utstyr}
		\label{equipment}
		\begin{tabular}{ll}
			\toprule
			magnesiumbånd & \\ 
			saltsyre & (\SI{6}{M}) \\
			smergelpapir & \\
			målerør & \\
			slangebit & \\
			hansker & \\
			spruteflaske & (ionebyttet vann) \\
			begerglass & (\SI{150}{mL}) \\
			kobbertråd & \\
			avbitertang & \\ 
			stativ med klemme & \\ 
			vekt & (felles) \\ \bottomrule 
		\end{tabular}
	\end{table}
	
	I tabell \ref{equipment} er nødvendig utstyr oppgitt.	Noter i journalen hva slags vekt du benytter og hvilken toleranse den har. Du bør bruke samme vekt til alle målingene.
	
	\subsubsection{Fremgangsmåte}
	\begin{itemize}
		\item Puss et magnesiumbånd med smergelpapir og tørk det for å fjerne mulig oksidlag.
		
		\item Klipp båndet i fem omtrent like store biter med avbitertang.
		
		\item Vei magnesiumbitene og notér vekten i journalen. Pass på at du har kontroll på hvilke biter som veier hva.
		
		\item Følgende del repeteres for hver magnesiumbit:
		\begin{itemize}
			\item Vikle den ene enden av kobbertråden rundt magnesiumbiten. 
			
			\item Hell ca. \SI{10}{mL} \SI{6}{M} saltsyre i målerøret. Fyll resten av røret forsiktig med ionebyttet vann slik at den tyngre saltsyren ligger i bunn med vannet oppå.
			
			\item Heng magnesiumbiten med kobbertråden ca. \SI{5}{cm} ned i målerøret og fest tråden ved å sette en slangebit i åpningen. Fyll opp med ionebyttet vann. Også hullet i slangebiten skal være fylt.
			
			\item Fyll begerglasset halvfullt med vann og sett opp et stativ med klemme ved siden av. Hold over åpningen i målerøret og snu det opp ned og sett den ned i begerglasset. Fest målerøret i stativet med klemmen.
			
			\item Saltsyren strømmer ned og reagere med magnesiumbiten så det produseres hydrogengass som samles i målerøret. Vent til reaksjonen er ferdig.
			
			\item Mål temperaturen i nærheten av målerøret, mål volumet av gassen (til nærmeste \SI{0.05}{mL}) og mål vannhøyden fra overflaten i begerglasset til bunnen av menisken i målerøret med en linjal.
		\end{itemize}
		
		\item En veileder vil oppgi barometertrykket.
		
	\end{itemize}	
	
	\clearpage
	
	\section{Rapport}
	Øverst på rapporten skal det stå:
	\begin{itemize}
		\item Dato for gjennomføring av labøvelsen.
		\item Navn og plassnummer på laboratoriet.
		\item Navn på labpartner.
		\item Gruppenummer.
		\item Gruppelærer.
	\end{itemize}
	
	Rapporten skal inneholde:
	
	\paragraph{Hensikt}
	Forklar \emph{kort} hensikten og målet med øvelsen med dine egne ord.
	
	\paragraph{Eksperimentelt}
	Forklar \emph{kort} den eksperimentelle delen med egne ord.
	
	\paragraph{Resultater}
	Oppsummer resultatene dine i en tabell som for hvert replikat angir
	\begin{itemize}
		\item innveid masse magnesium
		\item oppsamlet volum hydrogengass
		\item barometertrykket slik det er oppgitt på laboratoriet
		\item temperaturen der forsøket ble utført
		\item høyden på vannsøylen ved forsøkets slutt
		\item vannets damptrykk ved den målte temperaturen. Dette kan leses av fra Tabell \ref{vap} eller beregnes ved lineær interpolasjon.
		\item beregnet molar masse for magnesium
	\end{itemize}
	
	Vis detaljert utregning for replikat 1.
	
	Beregn gjennomsnittlig molar masse av magnesium, standardavvik og standardfeil. Bruk korrekt antall signifikante siffer.
	
	\paragraph{Diskusjon}
	\begin{itemize}
		\item Vurder resultatene dine.
		\item Hvordan stemmer den molare massen du har beregnet for magnesium med periodesystemets oppgitte verdi på \SI{24.31}{g/mol}.
		\item Skjedde det endringer eller feil under forsøket som kan ha påvirket resultatet?
	\end{itemize}

\end{document}